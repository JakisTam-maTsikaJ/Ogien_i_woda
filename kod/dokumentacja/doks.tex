\documentclass[10pt]{article}
\usepackage[utf8]{inputenc}
\usepackage[T1]{fontenc}

\title{Ogień i Woda}

\author{}
\date{}


\begin{document}
\maketitle

\section*{Technikalia}
Na grę składają się 3 główne projekty: menu, fizyka i mapy, gdzie każda z map dziedziczy po fizycje jako oddzielny projekt.


\section*{Menu}
\begin{itemize}
  \item z czego składa się menu?

  \item jak działa menu?

\end{itemize}

\section*{Fizyka}
\begin{itemize}
  \item jak działa fizyka?
\end{itemize}

\subsection*{Unikanie konfliktów między projektami}
Żeby fizyka można było dziedziczyć przez mapy trzeba usunąć część kodu, który tworzy CodeBlocks przy nowym projekcie. Celem jest uniknięcie konfliktów. Mowa o \texttt{build info} - całość trzeba było znaleźć w pliku \texttt{fizykaMain.cpp} i usunąć. Projekt działa w bez tego, a sam CodeBlocks później już tego nie dodaje.

\begin{verbatim}
wxString wxbuildinfo(wxbuildinfoformat format)
{
    wxString wxbuild(wxVERSION_STRING);

    if (format == long_f )
    {
#if defined(__WXMSW__)
        wxbuild << _T("-Windows");
#elif defined(__UNIX__)
        wxbuild << _T("-Linux");
#endif

#if wxUSE_UNICODE
        wxbuild << _T("-Unicode build");
#else
        wxbuild << _T("-ANSI build");
#endif // wxUSE_UNICODE
    }

    return wxbuild;
}
\end{verbatim}

\section*{Mapy}
\begin{itemize}
  \item jak działają mapy?
\end{itemize}

\subsection*{Tworzenie nowej mapy w CodeBlocks}
\begin{enumerate}
  \item Stwórz nowy projekt wxWidgets w CodeBlocks, przykładowo \texttt{mapa}
\end{enumerate}

\begin{enumerate}
  \setcounter{enumi}{1}
  \item Dodaj z pliku projektu z fizyką

  \begin{itemize}
    \item w opcjach projektu (w lewym pasku, prawy przycisk na ikonę projektu) za pomocą \texttt{Add files...} trzeba znaleźć (albo wrzucić do projektu) i dodać pliki \texttt{fizykaMain.cpp} i \texttt{fizykaMain.h}

    \item w opcjach projektu w \texttt{build options} trzeba dodać lokalizację projektu z fizyką

  \end{itemize}

  \item Ustal dziedziczenie po elementach \textit{fizyki} (tzn.: dodaj trochę kodu w klasach)

  \begin{itemize}
    \item dodaj dziedziczenie w \texttt{mapaMain.cpp} i użyj nowych funkcji:
  \end{itemize}\begin{verbatim}
MapaFrame::MapaFrame(wxWindow* parent,wxWindowID id) : fizykaFrame(parent, id) {

  ... // domyślny kod - nie trzeba nic zmieniać

  oczysc();
  n_graczy = 2;
  gracze = new wxStaticBitmap*[n_graczy] {
    ... // elementy graczy, np: StaticImage1, StaticImage2
  };

  n_przeszkod = ...;
  przeszkody = new wxStaticBitmap*[n_przeszkod] {
    ... // obrazki przeszkód, np: StaticImage3, StaticImage4, ...
  }
}
\end{verbatim}


  \begin{itemize}
    \item dodaj dziedziczenie i nadpisz funkcję, którą tworzy wxSmith w pliku \texttt{mapaMain.h}
  \end{itemize}\begin{verbatim}
#include "fizykaMain.h"
class mapkaFrame: public fizykaFrame {

  ... // właściwości klasy

  private: void Create(wxWindow*, wxWindowID, const wxChar*, wxPoint, wxSize, long style, const wxString&) {}
}
\end{verbatim}

Funkcja tworzy się domyślnie, jak używa się Smitha w CodeBlocksie, a jest przecież używana już w fizykaFrame, więc ten zapis sprawia, że nie jest uruchamiana 2 razy.

\end{enumerate}

\end{document}